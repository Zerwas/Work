\documentclass[]{scrartcl}

\usepackage[utf8]{inputenc}
\usepackage[german]{babel}
\usepackage{amsmath}
\usepackage{amsthm}
\usepackage{thmtools}
\usepackage{amssymb}
\usepackage{float}
\usepackage{tikz}
\usetikzlibrary{shapes.geometric}
%for \uwave
\usepackage[normalem]{ulem}
\usepackage{enumitem}
%\convertword{[.S ]}{.S}{.$S$} unfortunately no regular expressions
\usepackage{hyperref}
%for double [ bracket
\usepackage{stmaryrd}

\hypersetup{
	colorlinks,
	citecolor=black,
	linkcolor=black
}

\newcommand{\R}{\ensuremath{\mathbb{R}}}
\newcommand{\N}{\ensuremath{\mathbb{N}}}

\makeatletter
\newtheoremstyle{def}%
{\topsep}{\topsep}
{}{}
{\bfseries}{:}
{0.5em}
{\thmname{\@ifempty{#3}{#1}\@ifnotempty{#3}{#3}}}

\newtheoremstyle{satz}%
{\topsep}{\topsep}
{}{}
{\bfseries}{}
{0.5em}
{\thmname{Satz: \@ifempty{#3}{}\@ifnotempty{#3}{#3}}}
\makeatother

\declaretheorem[style=def,numbered=no]{definition}
\declaretheorem[style=satz,name=Satz,qed=\qedsymbol]{satz}



\begin{document}

\section*{Metrische R\"aume}
\begin{definition}[Metrischer Raum]
Sei $X$ eine Menge. Eine Metrik auf $X$ ist eine Abbildung $d:X\times X\to\left[0,\infty\right]$ mit
\begin{enumerate}
\item $\forall x,y\in X:d(x,y)=d(y,x)$
\item $\forall x,y,z\in X:d(x,z)\leq d(x,y)+d(y,z)$
\item $\forall x\in X:d(x,x)=0$
\item $\forall x,y\in X:d(x,y)=0\Rightarrow x=y$
\end{enumerate}
Das Paar $(X,d)$ ist ein metrischer Raum.
\end{definition}

\begin{definition}[Cauchy-Folge]
Sei $(X,d)$ ein metrischer Raum. Eine Folge $(a_n)_{n\in\N}$ in $X$ hei\ss t Cauchy-Folge
\[\Leftrightarrow\forall\varepsilon>0:\exists N\in\N:\forall n,m\geq N:d(a_n,a_m)<\varepsilon\]
\end{definition}

\begin{definition}[offen, abgeschlossen]
Sei $(X,d)$ ein metrischer Raum.

$U\subseteq X$ hei\ss t offen
\[\Leftrightarrow\forall x\in U:\exists s>0:B(x,s)\subseteq U.\]

$A\subseteq X$ hei\ss t abgeschlossen \[\Leftrightarrow X\setminus A\text{ offen.}\]
\end{definition}

\begin{satz}
Sei $(X,d)$ ein metrischer Raum. $A\subseteq X$ abgeschlossen
\[\Leftrightarrow\forall (a_n)_{n\in\N}\text{ in }A:a_n\to a\text{ in }X\Rightarrow x\in A\]
\end{satz}

\begin{definition}[vollst\"andig]
Sei $(X,d)$ ein metrischer Raum. $X$ hei\ss t vollst\"andig
\[\Leftrightarrow\text{jede Cauchy-Folge in $X$ konvergiert.}\]
\end{definition}

\section*{Stetigkeit}
\begin{definition}[Stetigkeit]
Seien $(X,d)$ und $(Y,e)$ metrische R\"aume, $f:X\to Y$, $a\in X$. Dann ist $f$ stetig in $a$
\begin{align*}
&\Leftrightarrow\forall\varepsilon>0:\exists\delta>0:\forall x\in X:d(a,x)<\delta\Rightarrow e(f(a),f(x))<\varepsilon.\\
&\Leftrightarrow\forall\varepsilon>0:\exists\delta>0:f(B_X(a,\delta))\subseteq B_Y(f(a),\varepsilon).
\end{align*}
\end{definition}

\begin{satz}[Zwischenwertsatz]
Seien $a,b\in\R$, $a<b$, $f:\left[a,b\right]\to\R$ stetig. Dann gilt:
\[\forall A\in\R:f(a)\leq A\leq f(b)\text{ oder } f(b)\leq A\leq f(a) \Rightarrow \exists x\in\left[a,b\right]:f(x)=A\]
\end{satz}

\begin{definition}[Folgenkompakt]
Sei $(X,d)$ ein metrischer Raum. $X$ hei\ss t folgenkompakt
\[\Leftrightarrow\text{Jede Folge in $X$ besitzt eine konvergente Teilfolge.}\]
\end{definition}

\begin{definition}[Lipschitz-Stetigkeit und gleichm\"a\ss ige Stetigkeit]
Seien $(X,d)$ und $(Y,e)$ metrische R\"aume, $f:X\to Y$, $a\in X$. Dann hei\ss t $f$

gleichm\"a\ss ig stetig
\[\Leftrightarrow\forall\varepsilon>0:\exists\delta>0:\forall x,x'\in X:d(x,x')<\delta\Rightarrow e(f(x),f(x'))<\varepsilon.\]

Lipschitz-stetig
\[\Leftrightarrow\exists L>0:\forall x,x'\in X:e(f(x),f(x'))\leq L\cdot d(x,x').\]
\end{definition}

\section*{Differentation}
\begin{definition}[Differenzierbar]
Sei $D\subseteq\R$, $f:D\to\R$, $x\in D$. $f$ hei\ss t in $x$ differezierbar
\[\Leftrightarrow\exists f'(x):=\lim\limits_{\xi\to x,\xi\in D\setminus\{x\}}\frac{f(\xi)-f(x)}{\xi-x}.\]
\end{definition}

\begin{satz}[von Rolle]
Sei $a<b$, $f:\left[a,b\right]\to\R$ stetig, $f(a)=f(b)=0$, $f$ in $(a,b)$ differenzierbar. Dann gilt:
\[\exists x\in(a,b):f'(x)=0.\]
\end{satz}

\begin{satz}[Mittelwertsatz]
Sei $a<b$, $f:\left[a,b\right]\to\R$ stetig, $f$ in $(a,b)$ differenzierbar. Dann gilt:
\[\exists x\in(a,b):f'(x)=\frac{f(b)-f(a)}{b-a}.\]
\end{satz}

%TODO integrale reihen
\section*{Topologie}
\begin{definition}[Inneres, Abschlu\ss, Rand]
Sei $(X,d)$ ein metrischer Raum. $Y\subseteq X$. Dann ist

das Innere von $Y$ gleich
\[\dot{Y}(=\text{int}(Y))=\{x\in Y\mid \exists\varepsilon>0:B(x,\varepsilon)\subseteq Y\}.\]

der Abschluss von $Y$ gleich
\[\overline{Y}(=\text{cl}(Y))=\{x\in X\mid \forall r>0:B(x,r)\cap Y\neq\emptyset\}.\]

der Rand von $Y$ gleich
\[\partial(Y)=\overline{Y}\setminus\dot{Y}.\]
\end{definition}

\begin{definition}[Kompaktheit]
$X$ hei\ss t kompakt
\[\Leftrightarrow\text{zu jeder offenen \"Uberdeckung $\mathcal{S}$ von $X$ gibt es $\mathcal{F}\subseteq\mathcal{S}$, $\mathcal{F}$ endlich, sodass $\bigcup\mathcal{F}=X$.}\]
\end{definition}

\begin{satz}[Kompaktheit und Folgenkompaktheit]
Sei $(X,d)$ ein metrischer Raum. Dann sind \"aquivalent:
\begin{enumerate}
\item $X$ ist kompakt.
\item $X$ ist folgenkompakt (i.e. jede Folge besitzt eine konvergente Teilfolge). 
\end{enumerate}
\end{satz}
\end{document}