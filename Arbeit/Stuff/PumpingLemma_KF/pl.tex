\documentclass[]{article}

\usepackage{amsthm}
\newtheorem{theorem}{Theorem}[section]
\newtheorem{claim}[theorem]{Claim}
\begin{document}
$$L=\left\lbrace a^ib^jc^k|i\neq j \wedge i\neq k \wedge j\neq k \right\rbrace$$
\begin{claim}
L ist nicht kontextfrei.
\end{claim}
\begin{proof}
Mit Hilfe des Ogden Lemma's.
$$w'=\underline{a^n}b^{n+n!}c^{n+2n!}$$
Alle a's sind markiert.
Es existiert keine Zerlegung $w'=uvwxy$, sodass in $vwx$ maximal n Buchstaben und in $vx$ mindestens ein Buchstabe markiert sind und $\forall i\geq0\ uv^iwx^iy\in L$.
\begin{itemize}
\item[Fall 1:] $v$ enth\"alt mindestens ein $a$ und mindestens einen weiteren anderen Buchstaben.\\ 
F\"ur $i=2$ ist das Wort $uv^2wx^2y$ offensichtlich nicht in $L$, da es nicht von der Form $a^*b^*c^*$ ist.
\item[Fall 2:] $x$ enth\"alt mindestens zwei unterschiedliche Buchstaben.\\ 
F\"ur $i=2$ ist das Wort $uv^2wx^2y$ offensichtlich nicht in $L$, da es nicht von der Form $a^*b^*c^*$ ist.
\item[Fall 3:] $v=a^r,x^s,r+s>0$ und $u,w,y$ beliebig.\\
Da $t=r+s\leq n$ muss t Teiler von $n!$ sein.
Daraus folgt das $i=1+\frac{n!}{t}$ eine ganze Zahl ist. 
$$v^ix^i=(a^{r})^{1+\frac{n!}{t}}(a^{s})^{1+\frac{n!}{t}}=a^{(r+s)(1+\frac{n!}{t})}=a^{t+n!}$$
$$uv^iwx^iy=a^{n+n!}b^{n+n!}c^{n+2n!}\notin L$$
\item[Fall 4:] $v=a^r,x=b^s,r>0,s>0$ und $u,w,y$ beliebig.\\
F\"ur $i=1+\frac{2n!}{r}$:
$$uv^iwx^iy=a^{n+2n!}b^{n+n!+s\frac{2n!}{r}}c^{n+2n!}\notin L$$
\item[Fall 5:] $v=a^r,x=c^s,r>0,s>0$ und $u,w,y$ beliebig.\\
F\"ur $i=1+\frac{n!}{r}$:
$$uv^iwx^iy=a^{n+n!}b^{n+n!}c^{n+2n!+s\frac{n!}{r}}\notin L$$
\end{itemize}
\end{proof}

\end{document}
