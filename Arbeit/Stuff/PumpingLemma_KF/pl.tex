\documentclass[]{article}

\usepackage{amsthm}
\usepackage{amssymb}
\usepackage{amsmath}
\newtheorem{theorem}{Theorem}[section]
\newtheorem{claim}[theorem]{Claim}
\begin{document}
$$L=\left\lbrace a^ib^jc^k|i\neq j \wedge i\neq k \wedge j\neq k \right\rbrace$$
\begin{claim}
L ist nicht kontextfrei.
\end{claim}
\begin{proof}
Mit Hilfe des Ogden Lemma's.
$$w'=\underline{a^n}b^{n+n!}c^{n+2n!}$$
Alle $a$'s sind markiert.
Es existiert keine Zerlegung $w'=uvwxy$, sodass in $vwx$ maximal $n$ Buchstaben und in $vx$ mindestens ein Buchstabe markiert sind und $\forall i\geq0\ uv^iwx^iy\in L$.
\begin{itemize}
\item[Fall 1:] $v$ enth\"alt mindestens ein $a$ und mindestens einen weiteren anderen Buchstaben.\\ 
F\"ur $i=2$ ist das Wort $uv^2wx^2y$ offensichtlich nicht in $L$, da es nicht von der Form $a^*b^*c^*$ ist.
\item[Fall 2:] $x$ enth\"alt mindestens zwei unterschiedliche Buchstaben.\\ 
F\"ur $i=2$ ist das Wort $uv^2wx^2y$ offensichtlich nicht in $L$, da es nicht von der Form $a^*b^*c^*$ ist.
\item[Fall 3:] $v=a^r,x=a^s,r+s>0$ und $u,w,y$ beliebig.\\
Da $t=r+s\leq n$ muss t Teiler von $n!$ sein.
Daraus folgt das $i=1+\frac{n!}{t}$ eine ganze Zahl ist. 
$$v^ix^i=(a^{r})^{1+\frac{n!}{t}}(a^{s})^{1+\frac{n!}{t}}=a^{(r+s)(1+\frac{n!}{t})}=a^{t+n!}$$
$$uv^iwx^iy=a^{n+n!}b^{n+n!}c^{n+2n!}\notin L$$
\item[Fall 4:] $v=a^r,x=b^s,r>0,s>0$ und $u,w,y$ beliebig.\\
F\"ur $i=1+\frac{2n!}{r}$:
$$uv^iwx^iy=a^{n+2n!}b^{n+n!+s\frac{2n!}{r}}c^{n+2n!}\notin L$$
\item[Fall 5:] $v=a^r,x=c^s,r>0,s>0$ und $u,w,y$ beliebig.\\
F\"ur $i=1+\frac{n!}{r}$:
$$uv^iwx^iy=a^{n+n!}b^{n+n!}c^{n+2n!+s\frac{n!}{r}}\notin L$$
\end{itemize}
\end{proof}
Warum funktioniert das einfache Pumping Lemma f\"ur kontextfreie Sprachen hier nicht?\\
$n=6$\\
$w'=a^jb^kc^l,j+k+l\geq6,$ O.B.d.A. $j<k<l$
\begin{itemize}
\item[Fall 1:]$k+1\leq l$\\
F\"ur die Zerlegung $u=a^jb^kc^{l-1},v=c,wxy=\epsilon$ gilt $uv^iwx^iy\in L$ f\"ur alle $i\geq0$, da:
$$uv^iwx^iy=a^jb^kc^{l+i-1}$$
\begin{align*}
j<k &\implies j \neq k\\
k<l-1\leq l+i-1 &\implies k\neq l \wedge j\neq k
\end{align*}
\item[Fall 2:]$k+1=l,j+1<k$\\
F\"ur die Zerlegung $u=a^jb^kc^{l-2},v=cc,wxy=\epsilon$ gilt $uv^iwx^iy\in L$ f\"ur alle $i\geq0$, da:
$$uv^iwx^iy=a^jb^kc^{l+2i-2}$$
F\"ur $i=0$ gilt:
\begin{align*}
l-2=k+1-2<k+1 &\implies k\neq l\\
j<k-1=l-2 &\implies j\neq l \wedge j\neq k
\end{align*}
F\"ur $i>0$ gilt:
\begin{align*}
j<k &\implies j \neq k\\
k=l-1<l+0\leq l+2i-2 &\implies k\neq l \wedge j\neq k
\end{align*}
\item[Fall 3:]$k+1=l, j+1=k$\\
F\"ur die Zerlegung $u=a^jb^kc^{l-3},v=ccc,wxy=\epsilon$ ($l\geq3$ da $n=6$) gilt $uv^iwx^iy\in L$ f\"ur alle $i\geq0$, da:
$$uv^iwx^iy=a^jb^kc^{l+3i-3}$$
F\"ur $i=0$ gilt:

\end{itemize}
\end{document}
