\documentclass[]{article}

\usepackage[utf8]{inputenc}
\usepackage[english]{babel}
\usepackage{amsthm}
\usepackage{amssymb}
\usepackage{amsmath}
\usepackage{minted}
\newtheorem{theorem}{Theorem}[section]
\newtheorem{lemma}[theorem]{Lemma}
\theoremstyle{definition}
\newtheorem{claim}[theorem]{Claim}


\begin{document}
\section{WFI or not?}
\begin{minted}{haskell}
take 0 xs     = []
take n []     = []
take n (x:xs) = [x]++(take (n-1) xs)
\end{minted}
\begin{minted}{haskell}
drop 0 xs     = xs
drop n []     = []
drop n (x:xs) = drop (n-1) xs
\end{minted}
\begin{claim}The following equation holds.
\begin{minted}{haskell}
(take n xs)++(drop n xs) = xs
\end{minted}
\end{claim}
\begin{proof} by induction over n.
\begin{itemize}
\item[BC:]n=0:
\begin{minted}{haskell}
  (take n xs)++(drop n xs) 
= (take 0 xs)++(drop 0 xs)
= []++xs
= xs
\end{minted}
\item[IH:]
\begin{minted}{haskell}
(take n' xs)++(drop n' xs) = xs
\end{minted}
holds for an arbitrary but fixed n' and all xs.
\item[IS:]n=n'+1
\subitem Case 1: xs=[]
\begin{minted}{haskell}
	  (take n xs)++(drop n xs)
	= (take (n'+1) [])++(drop (n'+1) []) 
	= []++[]
	= []
	= xs
\end{minted}
\subitem Case 2: xs=(x:xs')
\begin{minted}{haskell}
	  (take n xs)++(drop n xs)
	= (take (n'+1) (x:xs'))++(drop (n'+1) (x:xs'))
	= [x]++(take n' xs')++(drop n' xs')
	= [x]++xs'
	= xs
\end{minted}
\end{itemize}
\end{proof}
\newpage
\begin{minted}{haskell}
zip [] ys         = []
zip xs []         = []
zip (x:xs) (y:ys) = [(x,y)]++(zip xs ys)
\end{minted}
\begin{minted}{haskell}
length []     = 0
length (x:xs) = 1+(length xs)
\end{minted}
\begin{minted}{haskell}
min x y
   |x>y   = y
   |true  = x
\end{minted}
\begin{claim}\label{minZip}The following equation holds.
\begin{minted}{haskell}
length (zip xs ys) = min (length xs) (length ys)
\end{minted}
\end{claim}

\begin{proof} by structural induction over xs.
\begin{itemize}
\item[BC:]xs=[]
\begin{minted}{haskell}
  length (zip xs ys)
= length (zip [] ys)
= length []
= 0
= min 0 (length ys) --since (length ys) >= 0
= min (length xs) (length ys)
\end{minted}
\item[IH:]
\begin{minted}{haskell}
length (zip xs' ys) = min (length xs') (length ys)
\end{minted}
holds for an arbitrary but fixed xs' and all ys.
\item[IS:]xs=(x:xs')
\subitem Case 1: ys=[]
\begin{minted}{haskell}
	  length (zip xs ys)
	= length (zip (x:xs') []) 
	= length []
	= min (length (x:xs')) 0 --since (length (x:xs')) > 0
	= min (length xs) (length ys)
\end{minted}
\subitem Case 2: ys=(y:ys')
\begin{minted}{haskell}
	  length (zip xs ys)
	= length (zip (x:xs') (y:ys')) 
	= length ([(x,y)]++(zip xs' ys'))
	= 1+length (zip xs' ys')
	= 1+(min (length xs') (length ys'))
	= min (1+(length xs')) (1+(length ys'))
	= min (length (x:xs')) (length (y:ys'))
	= min (length xs) (length ys)
\end{minted}
\end{itemize}
\end{proof}
But now just for the fun of it let us proof Claim \ref{minZip} by well founded induction.
\begin{align*}
%([],ys)&\prec((x:xs),\hphantom{(y:\ }ys\hphantom{)})\\
%(xs,[])&\prec(\hphantom{(x:\ }xs\hphantom{)},(y:ys))\\
(xs,ys)&\prec((x:xs),(y:ys))
\end{align*}
We denote the transitive closure of $\prec$ by $<$. Obviously $<$ is a well-founded order.
\begin{proof} by well-founded induction with the predicate 
\begin{minted}{haskell}
P((xs,ys))=(length (zip xs ys) = min (length xs) (length ys))
\end{minted}.
%\iff intestead of = ?
\begin{itemize}
\item[] xs=[]: (Note that ([],ys) has no successors w.r.t. $<$.)
\begin{minted}{haskell}
  length (zip xs ys)
= length (zip [] ys)
= length []
= 0
= min 0 (length ys) --since (length ys) >= 0
= min (length xs) (length ys)
\end{minted}
\item[] ys=[]: This case can be treated analogously to xs=[].
\item[] xs=(x:xs'),ys=(y:ys)
\begin{minted}{haskell}
  length (zip xs ys)
= length (zip (x:xs') (y:ys')) 
= length ([(x,y)]++(zip xs' ys'))
= 1+(length (zip xs' ys'))
= 1+(min (length xs') (length ys')) {-since (xs',ys')<(xs,ys) 
                                         P((xs',ys')) holds-}
= min (1+(length xs')) (1+(length ys'))
= min (length (x:xs')) (length (y:ys'))
= min (length xs) (length ys)
\end{minted}
\end{itemize}
\end{proof}
\newpage
\subsection{Lexicographic Order}
\begin{minted}{haskell}
a 0 m = m+1
a n 0 = a (n-1) 1
a n m = a (n-1) (a n (m-1))
\end{minted}
\begin{minted}{haskell}
alist []     ys     = [1]++ys
alist (x:xs) []     = alist xs [x]
alist (x:xs) (y:ys) = alist xs (alist (x:xs) ys)
\end{minted}
\begin{claim}The following equation holds.
\begin{minted}{haskell}
length (alist xs ys) = a (length xs) (length ys)
\end{minted}
\end{claim}
\begin{align*}
(xs,ys)&\prec_l(\hphantom{(x:\ }xs\hphantom{)},(y:ys))\\
(xs,zs)&\prec_l((x:xs),\hphantom{(x:\ }ys\hphantom{)})
\end{align*}
We denote the transitive closure of $\prec_l$ by $<_l$. Obviously $<_l$ is a well-founded order.
\begin{proof} by well-founded induction with the predicate 
\begin{minted}{haskell}
P((xs,ys))=(length (alist xs ys) = a (length xs) (length ys))
\end{minted}
\begin{itemize}
\item[] xs=ys=[]: (Note that ([],[]) has no successors w.r.t. $<_l$.)
\begin{minted}{haskell}
  length (alist xs ys)
= length (alist [] [])
= length ([1]++[])
= 1
= a 0 0
= a (length []) (length [])
\end{minted}
\item[] xs=(x:xs'),ys=(y:ys)
\begin{minted}{haskell}
  length (alist xs ys)
= length (alist (x:xs') (y:ys')) 
= length (alist xs' (alist (x:xs') ys'))
= a (length xs') (length (alist (x:xs') ys')) 
                  {-since (xs',(alist (x:xs') ys'))<(xs,ys) 
                       P((xs',(alist (x:xs') ys'))) holds-}
= a (length xs') (a (length (x:xs')) (length ys'))
                  {-since (xs,ys')<(xs,ys) 
                        P((xs,ys') holds-}
= a (length xs') (a ((length xs')+1) (length ys'))
= a ((length xs')+1) ((length ys')+1)
= a (length (x:xs')) (length (y:ys'))
= a (length xs) (length ys)
\end{minted}
\end{itemize}
\end{proof}
\subsection{Multiset Order}
\begin{minted}{haskell}
r []     = []
r (x:xs) = r (xs++[(x+1)]++[(x+1)])
\end{minted}
\begin{claim}The following equation holds.
\begin{minted}{haskell}
r xs = []
\end{minted}
\end{claim}
%multiset order
\subsection{Arbitrary Order (Not ready just an idea)}
\begin{minted}{haskell}
f 0 = g 0
f a = b (g (r a))
\end{minted}
\begin{claim}The following equation holds.
\begin{minted}{haskell}
f a = g a
\end{minted}
\end{claim}
\end{document}
