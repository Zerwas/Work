\subsection{Lexicographic Order}
\begin{minted}{haskell}
a 0 m = m+1
a n 0 = a (n-1) 1
a n m = a (n-1) (a n (m-1))
\end{minted}
\begin{minted}{haskell}
alist []     ys     = [1]++ys
alist (x:xs) []     = alist xs [x]
alist (x:xs) (y:ys) = alist xs (alist (x:xs) ys)
\end{minted}
\begin{claim}The following equation holds.
\begin{minted}{haskell}
length (alist xs ys) = a (length xs) (length ys)
\end{minted}
\end{claim}
\begin{align*}
(xs,ys)&\prec_l(\hphantom{(x:\ }xs\hphantom{)},(y:ys))\\
(xs,zs)&\prec_l((x:xs),\hphantom{(x:\ }ys\hphantom{)})
\end{align*}
We denote the transitive closure of $\prec_l$ by $<_l$. Obviously $<_l$ is a well-founded order.
\begin{proof} by well-founded induction with the predicate 
\begin{minted}{haskell}
P((xs,ys))=(length (alist xs ys) = a (length xs) (length ys))
\end{minted}
\begin{itemize}
\item[] xs=ys=[]: (Note that ([],[]) has no successors w.r.t. $<_l$.)
\begin{minted}{haskell}
  length (alist xs ys)
= length (alist [] [])
= length ([1]++[])
= 1
= a 0 0
= a (length []) (length [])
\end{minted}
\item[] xs=(x:xs')
\subitem Case 1: ys=[]
\begin{minted}{haskell}
	  length (alist xs ys)
	= length (alist (x:xs') []) 
	= length (alist xs' [x])
	= a (length xs') (length [x]) 
              {-since (xs',[x])<_l(xs,ys) 
                      P((xs',[])) holds-}
	= a (length xs') 1
	= a ((length xs')+1) 0
	= a (length (x:xs')) (length [])
	= a (length xs) (length ys)
\end{minted}
\subitem Case 2: ys=(y:ys)
\begin{minted}{haskell}
	  length (alist xs ys)
	= length (alist (x:xs') (y:ys')) 
	= length (alist xs' (alist (x:xs') ys'))
	= a (length xs') (length (alist (x:xs') ys')) 
	                {-since (xs',(alist (x:xs') ys'))<_l(xs,ys) 
	                       P((xs',(alist (x:xs') ys'))) holds-}
	= a (length xs') (a (length (x:xs')) (length ys'))
	                {-since (xs,ys')<_l(xs,ys) 
	                        P((xs,ys') holds-}
	= a (length xs') (a ((length xs')+1) (length ys'))
	= a ((length xs')+1) ((length ys')+1)
	= a (length (x:xs')) (length (y:ys'))
	= a (length xs) (length ys)
\end{minted}
\end{itemize}
\end{proof}
\subsection{Multiset Order}
\begin{minted}{haskell}
r []     = []
r (0:xs) = r xs
r (x:xs) = r (xs++[(x-1)]++[(x-1)])
\end{minted}
\begin{claim}The following equation holds.
\begin{minted}{haskell}
r xs = []
\end{minted}
\end{claim}
%multiset order
