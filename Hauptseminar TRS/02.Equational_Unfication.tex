\section{Equational Unification}
In the following let $E$ be a set of identities of the form $\left\lbrace  e_1\approx f_1,\dots,e_n\approx f_n \right\rbrace$. Furthermore let $\mathcal{S}ig(E)$ denote the set of all function symbols occurring in $E$. 
\begin{definition}
An \textbf{$E$-unification Problem} over $\Sigma$ is a finite set $S$ of the form $S=\left\lbrace s_1\approxq_{E} t_1,\dots,s_n\approxq_{E} t_n\right\rbrace $ with $s_1,\dots,s_n,t_1,\dots,t_n \in T(\Sigma,V)$, $V$ being a countable set of Variables.\\
A substitution $\sigma$ is an \textbf{$E$-unifier} of $S$ iff $ \sigma(s_i)\approx_E \sigma(t_i)$ for all $1\leq i \leq n$.
The set of all $E$-unifiers of $S$ is denoted by $\mathcal{U}_E(S)$. $S$ is \textbf{$E$-unifiable} iff $\mathcal{U}_E(S)\neq \emptyset$.
\end{definition}
\begin{definition}
Let $S$ be an $E$-unification problem over $\Sigma$.
\begin{itemize}
\item $S$ is an \textbf{elementary} $E$-unification problem iff $\mathcal{S}ig(E)=\Sigma$.
\item $S$ is an $E$-unification problem \textbf{with constants} iff $\Sigma-\mathcal{S}ig(E)\subseteq\Sigma^{(0)}$
\item $S$ is an \textbf{general} $E$-unification problem iff $\Sigma-\mathcal{S}ig(E)$ contains an at least unary function symbol.
\end{itemize}
\end{definition}