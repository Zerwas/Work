\section{Boolean Rings}
%intro
\[B:=\left\lbrace 
\begin{aligned}
x+y&\approx y+x,& x*y&\approx y*x,\\
(x+y)+z&\approx x+(y+z),& (x*y)*z&\approx x*(y*z),\\
x+x&\approx 0,& x*x&\approx x,\\
0+x&\approx x,& 0*x&\approx 0,\\
x*(y+z)&\approx (x*y)+(x*z),& 1*x&\approx x
\end{aligned}
\right\rbrace \]
Since $+$ and $*$ are associative we can omit most of the brackets. Furthermore we often write $xy$ instead of $x*y$.
Lets consider a semantic interpretation of $B$ the two element boolean ring $\mathcal{B}_2$ with the carrier set \textbf{2}$:=\left\lbrace 0,1\right\rbrace$ where $*$ is "and" and $+$ is "exclusive or". 
%B_2
To see that there are other models of $B$ lets consider the powerset interpretation $\mathcal{P}_S$:
\begin{align*}
\Delta^{\mathcal{P}_S}&:=2^S\\
(x+y)^{\mathcal{P}_S}&:=x^{\mathcal{P}_S}\Delta y^{\mathcal{P}_S}& (x*y)&:=x^{\mathcal{P}_S}\cap y^{\mathcal{P}_S}\\
0^{\mathcal{P}_S}&:=\emptyset& 1^{\mathcal{P}_S}&:=S
\end{align*}.
Where $x\Delta y:=(x\setminus y)\cup(y\setminus x)$ is the symmetric difference of x and y. It is easy to see why $\mathcal{P}_S$ is a model of $B$. Lets just consider distributivity in a bit more detail.
\begin{center}
\def\CircleX{(0.5,0.866) circle (0.8)}
\def\CircleY{(0,0) circle (0.8)}
\def\CircleZ{(1,0) circle (0.8)}
\begin{tikzpicture}[fill=gray]
% left hand
\begin{scope}
\clip (-2,-2) rectangle (2,2)
      \CircleZ;
\fill \CircleY;
\end{scope}
% right hand
\begin{scope}
\clip (-2,-2) rectangle (2,2)
      \CircleY;
\fill \CircleZ;
\end{scope}
\begin{scope}
      \clip \CircleY
      		\CircleZ;
      \fill[red] \CircleX;
\end{scope}
\begin{scope}
      \clip \CircleY
      \fill[black] \CircleZ;
\end{scope}
% outline
\draw \CircleX  node {$x$}
      \CircleY  node {$y$}
      \CircleZ  node {$z$}
      (-2,-1.5) rectangle (3,2.366) node [text=black,above] {$S$};
\end{tikzpicture}
\end{center}