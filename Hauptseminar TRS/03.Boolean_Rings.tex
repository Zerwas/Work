\section{Boolean Rings}
%TODO intro
\[B:=\left\lbrace 
	\begin{aligned}
		x+y     & \approx y+x,         & x*y     & \approx y*x,     \\
		(x+y)+z & \approx x+(y+z),     & (x*y)*z & \approx x*(y*z), \\
		x+x     & \approx 0,           & x*x     & \approx x,       \\
		0+x     & \approx x,           & 0*x     & \approx 0,       \\
		x*(y+z) & \approx (x*y)+(x*z), & 1*x     & \approx x        
	\end{aligned}
	\right\rbrace \]
	Since $+$ and $*$ are associative we can omit most of the brackets. Furthermore we often write $xy$ instead of $x*y$.
	Lets consider an interpretation of $B$ the two element boolean ring $\BTwo$ with the carrier set $\textbf{2}:=\left\lbrace\bot,\top\right\rbrace$ where $*$ is \textquotedblleft and\textquotedblright\ and $+$ is \textquotedblleft exclusive or\textquotedblright:
	\begin{align*}
		(x+y)^\BTwo & :=\left( x^\BTwo\wedge\neg y^\BTwo\right)  \vee \left( \neg x^\BTwo\wedge y^\BTwo\right) & (x*y)^\BTwo & :=x^\BTwo\wedge y^\BTwo \\
		0^\BTwo     & :=\bot                                                                                      & 1^\BTwo     & :=\top                
	\end{align*}
	%TODO exercise 10.26
	It is easy to see that $\BTwo$ is indeed a model of $B$. Furthermore we can transform a term back from an Boolean algebra into Boolean ring theory:
	\begin{align*}
		x \wedge y & \mapsto x*y     \\
		x\vee y    & \mapsto x+y+x*y \\
		\neg x     & \mapsto x+1
	\end{align*}
	We work with + and * instead of $\vee,\wedge$ and $\neg$ because in Boolean rings we have a very convenient normal form which makes the following proofs easier.\\
	% maybe rephrase
	Lets consider another model of $B$ the powerset interpretation $\PSet$ with the carrier set $2^S$:
	\begin{align*}
		(x+y)^\PSet & :=x^\PSet\Delta y^\PSet & (x*y)^\PSet & :=x^\PSet\cap y^\PSet \\
		0^\PSet     & :=\emptyset             & 1^\PSet     & :=S                   
	\end{align*}
	Where $x\Delta y:=(x\setminus y)\cup(y\setminus x)$ is the symmetric difference of x and y. It is easy to see why $\PSet$ is a model of $B$. Lets just consider distributivity in detail.
	\begin{align*}
		\left( x*\left( y+z\right) \right) ^\PSet & =x^\PSet\cap\left( y^\PSet\Delta z^\PSet\right)                                                                                                                                                \\
		                                          & =x^\PSet\cap\left( \left( y^\PSet\setminus z^\PSet\right) \cup\left( z^\PSet\setminus y^\PSet\right) \right)                                                                                   \\
		                                          & =\left( \left( x^\PSet\cap y^\PSet\right) \setminus z^\PSet\right) \cup\left( \left( x^\PSet\cap z^\PSet\right) \setminus y^\PSet\right)                                                       \\
		                                          & =\left( \left( x^\PSet\cap y^\PSet\right) \setminus \left( x^\PSet\cap z^\PSet\right) \right) \cup\left( \left( x^\PSet\cap z^\PSet\right) \setminus \left( x^\PSet\cap y^\PSet\right) \right) \\
		                                          & =\left( x^\PSet\cap y^\PSet\right) \Delta \left( x^\PSet\cap z^\PSet\right)                                                                                                                    \\
		                                          & =\left( \left( x*y\right) +\left( x*z\right)\right)^\PSet                                                                                                                                      
	\end{align*}
	Here is a less formal explanation for why this identity holds in $\PSet$.\\
	\def\f{1.3}
	\def\CircleX{(\f*0.5,\f*0.866) circle (\f*0.8)}
	\def\CircleY{(\f*0,0) circle (\f*0.8)}
	\def\CircleZ{(\f*1,0) circle (\f*0.8)}
	\begin{tabular*}{\textwidth}{cc}
		\begin{tikzpicture}[fill=gray]
			\begin{scope}
				\clip \CircleY
				\CircleZ;
				\fill \CircleX;
			\end{scope}
			% left hand
			\begin{scope}
				\clip (-2,-1.5) rectangle (3,2.366)
				\CircleZ;
				\fill[pattern=north east lines] \CircleY;
			\end{scope}
			% right hand
			\begin{scope}
				\clip (-2,-1.5) rectangle (3,2.366)
				\CircleY;
				\fill[pattern=north east lines] \CircleZ;
			\end{scope}
			% fill y \cap z white
			\begin{scope}
				\clip \CircleY;
				\fill[white] \CircleZ;
			\end{scope}
			% draw circles
			\draw \CircleX  node [above] {$x^\PSet$}
			\CircleY  node [left] {$y^\PSet$}
			\CircleZ  node [right] {$z^\PSet$}
			(-2,-1.5) rectangle (3,2.366) (-2,2.366) node [below right] {$S$}
			(-2,-1.5) node [below right] {$(x*(y+z))^\PSet$};
		\end{tikzpicture}
		&
		\begin{tikzpicture}[fill=gray]
			\begin{scope}
				\clip \CircleY
				\CircleZ;
				\fill \CircleX;
			\end{scope}
			% fill y \cap z white
			\begin{scope}
				\clip \CircleY;
				\fill[white] \CircleZ;
			\end{scope}
			% fill z /cap x 
			\begin{scope}
				\clip \CircleX;
				\fill[pattern=north west lines] \CircleZ;
			\end{scope}
			% fill y /cap x 
			\begin{scope}
				\clip \CircleX;
				\fill[pattern=north east lines] \CircleY;
			\end{scope}
			% draw circles
			\draw \CircleX  node [above] {$x^\PSet$}
			\CircleY  node [left] {$y^\PSet$}
			\CircleZ  node [right] {$z^\PSet$}
			(-2,-1.5) rectangle (3,2.366) (-2,2.366) node [below right] {$S$}
			(-2,-1.5) node [below right] {$((x*y)+(x*z))^\PSet$};;
		\end{tikzpicture}
	\end{tabular*}
	This small example should just show that there are other models of $B$ with rather common interpretations of + and * apart from $\BTwo$. Note that if $|S|=1$ then $\PSet$ and $\BTwo$ are isomorphic.
	%correct word?
	%TODO we only consider terms over 0,1,+,*
	\subsection{Polynomials}
	\begin{definition}
		A product of distinct variables is a \textbf{monomial} (e.g.$\ xyz$). And a sum of distinct monomials is a \textbf{polynomial} (e.g.$\ x+xy+yz$).
	\end{definition}
	We compare monomials and polynomials modulo commutativity and associativity. So two monomials are distinct iff the sets of variables occurring in them are distinct and two polynomials are distinct iff the sets of their monomials are distinct.
	Here are some examples for clarification:
	\begin{align*}
		yxz & =zyx     & xy+yz & =zy+xy  \\
		yx  & \neq yxz & xy+yz & \neq xy 
	\end{align*}
	Note that we did not introduce a new symbol for equality of polynomials and just use the same as for syntactic equality since it will be clear from the context which one we mean. Now we can transform every term over $\left\lbrace 0,1,+,*\right\rbrace$ 
	%TODO right order?
	 into a (w.r.t. equality of polynomials) unique $\approx_B$-equivalent polynomial, its \textbf{polynomial form}. Since 1 is the neutral element of * we write 1 for the polynomial containing only the empty monomial correspondingly we identify 0 with the empty polynomial. Now the polynomial form can be computed recursively as follows:\\
	\begin{description}
		\item[$x,0,1:$] This is the base case, variables and the constants 0 and 1 are already polynomials.
		\item[$t_1+t_2:$] Let $p_1$ and $p_2$ be the polynomial forms of $t_1$ and $t_2$ the polynomial form of $t_1+t_2$ is obtained by removing all pairs of equivalent monomials from $p_1+p_2$. Since we have $\left\lbrace0+x\approx x,x+x\approx0\right\rbrace\in B$  this rule preserves $\approx_B$-equivalence.
		%explain 0?
		\item[$t_1*t_2:$]  Let $p_1=m_1+\dots+m_k$ and $p_2=n_1+\dots+n_l$ be the polynomial forms of $t_1$ and $t_2$. The polynomial form of $t_1*t_2$ is obtained by removing all pairs of equivalent monomials from $p_1*p_2$ which when multiplied out is the sum
		\begin{align*}
			m_1*n_1+\dots+m_1*n_l+\dots+m_k*n_1+\dots+m_k*n_l 
		\end{align*} 
		where the product of two monomials $m=x_1\dots x_r$ and $n=y_1\dots y_s$ is the monomial obtained by removing repeated occurrences of the same variable from $x_1\dots x_r y_1\dots y_s$. Since we have $\left\lbrace x*x\approx x\right\rbrace\in B$  this rule preserves $\approx_E$-equivalence. Note that if $t_1=1$ then we multiply every monomial in $p_2$ with the empty monomial which does not change anything so the result is as expected just $p_2$.
	\end{description}
	The polynomial form of $t$ is denoted by $t{\downarrow_P}$. It is easy to see that all rules preserve $\approx_B$-equivalence and hence $t\approx_Bt{\downarrow_P}$.
	%example?
	\begin{theorem}\label{basicBR}
		The following statements are equivalent:
		\begin{enumerate}
			\item $\BTwo\models s\approx t,$
			\item $s{\downarrow_P}=t{\downarrow_P},$
			\item $s\approx_B t.$
		\end{enumerate}
	\end{theorem}
	We will not proof this theorem but instead look at a bigger example. Let $s:=(y+1)*(x+y)+(y+1)*x$ and $t:=0$ and go through 1, 2 and 3 from theorem \ref{basicBR}.
	\begin{enumerate}
			\item $\BTwo$ is a model of $s\approx t$.
			\begin{align*}
				s^\BTwo & =\left( \left( y+1\right)\left( x+y\right)+\left( y+1\right)x\right)^\BTwo       \\
				& =\left( s_1^\BTwo\wedge \neg s_2^\BTwo\right)\vee\left( \neg s_1^\BTwo\wedge s_2^\BTwo\right) \\
				&\hspace{8pt}\begin{aligned}
				s_1 & =\left( y+1\right)^\BTwo\wedge\left( x+y\right)^\BTwo & s_2 & =   \left( y+1\right)^\BTwo\wedge x^\BTwo \\
				    & =\neg y^\BTwo\wedge\left( x+y\right)^\BTwo            &     & = \neg y^\BTwo\wedge x^\BTwo              \\
				&=\neg y^\BTwo\wedge\left( \left( x^\BTwo\wedge\neg y^\BTwo\right)\vee\left( \neg x^\BTwo\wedge y^\BTwo\right)\right)\\
				& = x^\BTwo\wedge\neg y^\BTwo
				\end{aligned}\\
				s^\BTwo & =\left( \left( x^\BTwo\wedge\neg y^\BTwo\right)\wedge \neg \left( \neg y^\BTwo\wedge x^\BTwo\right)\right)\vee\left( \neg\left( x^\BTwo\wedge\neg y^\BTwo\right)\wedge \left( \neg y^\BTwo\wedge x^\BTwo\right)\right) \\
				& =\left( x^\BTwo\wedge\left( \neg y^\BTwo\wedge \left( y^\BTwo\vee\neg x^\BTwo\right)\right)\right)\vee\left( \left( \left( \neg x^\BTwo\vee y^\BTwo\right)\wedge \neg y^\BTwo\right)\wedge x^\BTwo\right)              \\
				& =\left( x^\BTwo\wedge\neg y^\BTwo\wedge \neg x^\BTwo\right)\vee\left( \neg x^\BTwo\wedge \neg y^\BTwo\wedge x^\BTwo\right)                                                                                             \\
				& =\bot\vee\bot\\
				& =t^\BTwo 
			\end{align*}
			\item The polynomial forms $s{\downarrow_P}$ and $t{\downarrow_P}$ are equal.
			\begin{align*}
				s & \approx_B (y+1)(x+y)+(y+1)x \\
				  & \approx_B yx+yy+x+y+yx+x    \\
				  & \approx_B yx+yx+y+y+x+x     \\
				  & \approx_B 0
			\end{align*}
			\begin{align*}
				s{\downarrow_P}=0=t{\downarrow_P}
			\end{align*}
			\item $s\approx_B t$ is a consequence of $B$.
			\begin{align*}
				s & =\ \ (y+1)(x+y)+(y+1)x \\
				  & \approx_B (y+1)(x+y+x)      \\
				  & \approx_B y+y               \\
				  & \approx_B 0                 \\
				  & =\ \ t                 
			\end{align*}
		\end{enumerate}
		A nice consequence from theorem \ref{basicBR} is that $\approx_B$ is decidable, because we could just compare the computable polynomial forms, or test semantic equality in $\BTwo$.
		\subsection{Unification}
		Note that until now we have not said anything about unification we just introduced the equational theory $B$, the semantic interpretation $\BTwo$ and showed some properties of the word problem in $\approx_B$. Now we will look at some ways to find most general $B$-unifiers of $B$-unification problems. But fist we show that unification modulo $\approx_B$ is closely related to equation solving in $\BTwo$.
		\begin{lemma}\mbox{}
		\begin{enumerate}
		\item Every solution of $s\approxq_B t$ in $\BTwo$ can be viewed as a $B$-unifier.
		\item If $s\approxq_B t$ has a $B$-unifier then $s\approxq t$ has a solution in $\BTwo$.
		\end{enumerate}
		\end{lemma}
		%TODO before or after proof? Note that we did not use iff because not every unifier can be viewed as a solution in $\BTwo$. example
		\begin{proof}\mbox{}
		\begin{itemize}
		\item[(1)]Let $\varphi:V\mapsto\textbf{2}$ be a solution of $s\approxq t$ in $\BTwo$ and $\widehat{\varphi}$ the homeomorphic extension of $\varphi$, i.e. $\widehat{\varphi}(s)=\widehat{\varphi}(t)$. From $\varphi$ we now can define a mapping $\phi:V\mapsto \TermSet$ to ground terms in the following way:
		\begin{align*}
		\phi(x):=\begin{cases}
		1 & \text{if }\varphi(x)=\top\\
		0 & \text{if }\varphi(x)=\bot
		\end{cases}
		\end{align*}
		for all $x\in V$. Lets denote the homeomorphic extension of $\phi$ by $\widehat{\phi}$. Let $\psi:\TermAlgebra\mapsto\BTwo$ be an arbitrary homomorphism since $\widehat{\phi}$ is a mapping to ground terms $\psi\widehat{\phi}=\widehat{\varphi}$. So the following
		\begin{align*}
		\psi\left( \widehat{\phi}(s)\right)=\widehat{\varphi}(s)=\widehat{\varphi}(t)=\psi\left( \widehat{\phi}(t)\right) 
		\end{align*}
		holds for all homomorphisms $\psi$ and hence $\BTwo\models\widehat{\phi}(s)\approx\widehat{\phi}(t)$. Theorem \ref{basicBR} yields $\widehat{\phi}(s)\approx_B\widehat{\phi}(t)$, i.e. the mapping $\phi$ (which we got directly from the solution $\varphi$), when viewed as a substitution, is a $B$-unifier.
		%TODO replace phi
		\item[(2)] Let $\sigma$ be a $B$-unifier of $s\approxq_B t$, i.e. $\sigma(s)\approx_B\sigma(t)$. Now theorem \ref{basicBR} yields $\BTwo\models\sigma(s)\approx\sigma(t)$ and hence $\psi\left(\sigma (s)\right)=\psi\left(\sigma (t)\right)$ for all homomorphisms $\psi:\TermAlgebra\mapsto\BTwo$. Hence $\psi\sigma:V\mapsto\textbf{2}$ is a solution in $\BTwo$.
		%TODO \psi\sigma ok? V->\TermSet=(?)\TermAlgebra->2
		\end{itemize}
		\end{proof}
		%TODO more text?
		%example
		Lets consider the B-unification problem $x+y+z\approxq z+1$ as small example for clarification.
		\begin{itemize}
		\item[(1)]The mapping 
		\begin{align*}
		\varphi(w):=\begin{cases}
		\bot& \text{if }w=x\\
		\top& \text{if }w\neq x
		\end{cases}
		\end{align*}is a solution of our problem. The substitution $\sigma:=\left\lbrace x\mapsto0,y\mapsto1,z\mapsto1\right\rbrace$ is $\varphi$ viewed as a $B$-unifier.
		\item[(2)]The substitution $\sigma':=\left\lbrace y\mapsto x+1,z\mapsto 1\right\rbrace $ obviously is a $B$-unifier. Let $\varphi$ be a mapping with $\varphi(x):=\top$ for all $x\in V$ and $\widehat{\varphi}$ its homeomorphic extension.
		\begin{align*}
		\widehat{\varphi}\left(\sigma'(w)\right)=\begin{cases}
		(\top+1)^\BTwo=\bot& \text{if } w=y\\
		1^\BTwo=\top& \text{if } w=z\\
		\top&  \text{otherwise}
		\end{cases}
		\end{align*}
		is a solution in $\BTwo$.
		\end{itemize}
		%TODO intro unitary,t\approx0
		\subsection{Successive variable elimination}
		The idea of this unification algorithm is %TODO intro
		
		Every term $t$ can be written as $x*r+(x+1)*s$ such that $x\notin\mathcal{V}ar(r,s)\subset\mathcal{V}ar(t)$. For example, we can split the polynomial form of $t$ into two sets of monomials, those who contain $x$  $\left(l_x\right)$ and those who do not contain $x$ $(s)$. Now $l$ is obtained by removing every occurrence of $x$ in $l_x$ and $r:=l+s$. To see why this works consider the small example $t:=yx+z$. Now $r:=y+z$ and $s:=z$.
		\begin{align*}
		x*(y+z)+(x+1)*z&\approx_B xy+xz+xz+z\\
		&\approx_B xy+z\\
		&\approx_B t
		\end{align*}
		This simple observation is the basis of successive variable elimination.
		Before we come to the main theorem of successive variable elimination we introduce a strong type of most general $B$-unifiers the reproductive $B$-unifiers.
		\begin{definition}
		A $E$-unifier $\sigma$ of a $B$-unification problem $S$ is a \textbf{reproductive $E$-unifier} iff $\tau(\sigma(x))\approx_E\tau(x)$ for every unifier $\tau$ of $S$ and every $x$.
		\end{definition}
		Note that for a normal mgu $\sigma$ of a $E$-unification problem $S$ it only has to hold that for every $E$-unifier $\tau$ of $S$ there has to exist a substitution $\theta$ such that $\theta(\sigma(x))\approx_E\tau(x)$ for every $x\in X$. 
		%TODO more explanation?
		\begin{theorem}\label{sucVEli}
		Let $t\approx_B x*r+(x+1)*s$ such that $x\notin\mathcal{V}ar(r,s)\subset\mathcal{V}ar(t)$ and define $t':=r*s$.
		\begin{enumerate}
		\item Every $B$-unifier of $t\approxq_B0$ is a $B$-unifier of $t'\approxq_B0$.
		%reproductive unifier->mgu
		\item If $\sigma$ is a reproductive $B$-unifier of $t'\approxq_B0$ and $x\notin\mathcal{D}om(\sigma)$, then 
		\begin{align*}
		\sigma':=\sigma\cup\left\lbrace x\mapsto x*(\sigma(r)+\sigma(s)+1)+\sigma(s)\right\rbrace
		\end{align*}
		is a reproductive $B$-unifier of $t\approxq_B0$.
		\end{enumerate}
		\end{theorem}
		\begin{proof}\mbox{}
		\begin{itemize}
		\item[(1)] Let $\tau$ be a $B$-unifier of $t\approxq_B0$ and hence $\tau(t)\approx_B0$.
		\begin{align*}
		&&\tau(x)*\tau(r)+(\tau(x)+1)*\tau(s)&\approx_B0\\
		&\iff&\tau(x)*\tau(r)*\tau(s)+(\tau(x)+1)*\tau(s)*\tau(r)&\approx_B0*\tau(s)*\tau(r)\\
		&\iff&\tau(x)*\tau(r*s)+(\tau(x)+1)*\tau(s*r)&\approx_B0\\
		&\iff&(\tau(x)+\tau(x)+1)*\tau(s*r)&\approx_B0\\
		&\iff&\tau(s*r)&\approx_B0\\
		\end{align*}
		So $\tau$ is also a $B$-unifier of $t'\approxq_B0$.
		\item[(2)] Let $\sigma$ is a reproductive $B$-unifier of $t'\approxq_B0$ and $x\notin\mathcal{D}om(\sigma)$. It is easy to show that $\sigma'$ is a $B$-unifier of $t\approxq_B0$:
		\begin{align*}
		\sigma'(t)&\approx_B\sigma'(x)*\sigma'(r)+(\sigma'(x)+1)*\sigma'(s)\\
		&=\ \ (x*(\sigma(r)+\sigma(s)+1)+\sigma(s))*\sigma(r)+(\sigma'(x)+1)*\sigma(s)\\
		&\approx_B(x*(\sigma(r)+\sigma(s)*\sigma(r)+\sigma(r))+\sigma(s)*\sigma(r))+(\sigma'(x)+1)*\sigma(s)\\
		&\approx_B(x*(0+\sigma(s*r))+\sigma(s*r))+(\sigma'(x)+1)*\sigma(s)\\
		&\approx_B(x*\sigma(t')+\sigma(t'))+(\sigma'(x)+1)*\sigma(s)\\
		&\approx_B0+(x*(\sigma(r)+\sigma(s)+1)+\sigma(s)+1)*\sigma(s)\\
		&\approx_B(x*(\sigma(r)*\sigma(s)+\sigma(s)+\sigma(s))+\sigma(s)+\sigma(s))\\
		&\approx_B(x*(\sigma(r*s))+0)\\
		&\approx_B0\\
		\end{align*}
		Now we show that $\sigma'$ is also reproductive. Let $\tau$ be a $B$-unifier of $t\approxq_B0$. Because $\sigma$ is a reproductive $B$-unifier of $t'\approxq_B0$ and (1) implies that $\tau$ is indeed a $B$-unifier of $t'\approxq_B0$ it follows that $\tau(\sigma(y))\approx_B\tau(y)$ for all $y$. Therefore $\tau(\sigma'(y))=\tau(\sigma(y))\approx_B\tau(y)$ for all $y\neq x$. For $y=x$:
		\begin{align*}
		\tau(\sigma'(x))&=\ \ \tau(x*(\sigma(r)+\sigma(s)+1)+\sigma(s))\\
		&\approx_B \tau(x)*(\tau(\sigma(r))+\tau(\sigma(s))+1)+\tau(\sigma(s))\\
		&\approx_B \tau(x)*(\tau(r)+\tau(s)+1)+\tau(s)\\
		&\approx_B \tau(x)*\tau(r)+\tau(x)*\tau(s)+\tau(x)+\tau(s)\\
		&\approx_B \tau(x)*\tau(r)+(\tau(x)+1)*\tau(s)+\tau(x)\\
		&\approx_B \tau(t)+\tau(x)\\
		&\approx_B \tau(x)\\
		\end{align*}
		\end{itemize}
		\end{proof}
		Now lets consider as example the $B$-unification problem $xy+yz+xz+1\approxq_B0$. First we eliminate $x$.
		\begin{align*}
		& x*(y+z+yz+1)+(x+1)*(yz+1)\approxq_B0& r=y+z+yz+1,s=yz+1\\
		& r*s\approx_Byz+yz+yz+yz+y+z+yz+1\approx_By+z+yz+1
		\end{align*}
		Now we eliminate y.
		\begin{align*}
		& y*(1+z+z+1)+(y+1)*(z+1)\approxq_B0& r'=1+z+z+1,s'=z+1\\
		& r'*s'\approx_B(1+z+z+1)*(z+1)\approx_B0*(z+1)\approx_B0
		\end{align*}
		Obviously $0\approxq0$ is $B$-unifiable with the reproductive $B$-unifier $\sigma''$ which is the identity on all variables. From theorem \ref{sucVEli} it follows that $\sigma'$ defined by
		\begin{align*}
		\sigma'&:=\sigma''\cup\left\lbrace y\mapsto y*(\sigma''(r')+\sigma''(s')+1)+\sigma''(s')\right\rbrace\\
		\sigma'&:=\sigma''\cup\left\lbrace y\mapsto y*(0+z+1+1)+z+1\right\rbrace\\
		\sigma'&:=\sigma''\cup\left\lbrace y\mapsto yz+z+1\right\rbrace
		\end{align*}
		is a reproductive $B$-unifier of $y*(1+z+z+1)+(y+1)*(z+1)\approxq_B0$. Correspondingly $\sigma$ defined by
		\begin{align*}
		\sigma&:=\sigma'\cup\left\lbrace x\mapsto x*(\sigma'(r+s)+1)+\sigma'(s)\right\rbrace\\
		\sigma&:=\sigma'\cup\left\lbrace x\mapsto x*(\sigma'(y+z)+1)+(yz+z+1)*z+1\right\rbrace\\
		\sigma&:=\sigma'\cup\left\lbrace x\mapsto x*(yz+z+1+z+1)+yz+1\right\rbrace\\
		\sigma&:=\sigma'\cup\left\lbrace x\mapsto xyz+yz+1\right\rbrace\\
		\end{align*}
		is a reproductive $B$-unifier of $x*(y+z+yz+1)+(x+1)*(yz+1)\approxq_B0$ and also of our original problem $xy+yz+xz+1\approxq_B0$.
		%TODO outtro
		\subsection{Löwenheim's formula}
		This algorithm is not as intuitive as successive variable elimination but much more interesting. The idea is to turn an arbitrary $B$-unifier $\tau$ of $t\approxq_B0$ into an mgu (even a reproductive $B$-unifier).
		\begin{theorem}\label{lowenheim}
		Let $\tau$ be a $B$-unifier of $t\approxq_B0$. The substitution $\sigma$ defined by
		\begin{align*}
		\sigma(x):=\begin{cases}
		(t+1)*x+t*\tau(x) &\text{if }x\in\mathcal{V}ar(t) \\
		x &\text{if }x\notin\mathcal{V}ar(t)
		\end{cases}
		\end{align*}
		is a reproductive $B$-unifier of $t\approxq_B0$.
		\end{theorem}
		Before we can proof this theorem we need the following lemma to proof that $\sigma$ actually is a $B$-unifier. 
		\begin{lemma}\label{lowenLemma}
		If $\sigma(x)=(s+1)*\sigma_1(x)+s*\sigma_2(x)$ for all $x\in\mathcal{V}ar(t)$, then $\sigma(t)=(s+1)*\sigma_1(t)+s*\sigma_2(t)$.
		\end{lemma}
		\begin{proof}
		We show this by structural induction on $t$.
		\begin{description}
		\item[$t=x:$]The base case is trivial:
				\begin{align*}
				\sigma(t)&=(s+1)*\sigma_1(x)+s*\sigma_2(x)\\
				&=(s+1)*\sigma_1(t)+s*\sigma_2(t)
				\end{align*}
		\item[$t=0:$]
				\begin{align*}
				\sigma(t)&=0
				\approx_B0+0
				\approx_B(s+1)*0+s*0\\
				&=(s+1)*\sigma_1(t)+s*\sigma_2(t)
				\end{align*}
		\item[$t=1:$]
				\begin{align*}
				\sigma(t)&=1
				\approx_Bs+1+s
				\approx_B(s+1)*1+s*1\\
				&=(s+1)*\sigma_1(t)+s*\sigma_2(t)
				\end{align*}
		\item[$t=t_1+t_2:$]Using the induction hypothesis we obtain:
				\begin{align*}
				\sigma(t)&=\ \ \sigma(t_1)+\sigma(t_2)\\
				&\approx_B(s+1)*\sigma_1(t_1)+s*\sigma_2(t_1)+(s+1)*\sigma_1(t_2)+s*\sigma_2(t_2)\\
				&\approx_B(s+1)*(\sigma_1(t_1)+\sigma_1(t_2))+s*(\sigma_2(t_1)+\sigma_2(t_2))\\
				&=\ \ (s+1)*\sigma_1(t)+s*\sigma_2(t)\\
				\end{align*}
		\item[$t=t_1*t_2:$]Using the induction hypothesis we obtain:
				\begin{align*}
				\sigma(t)&=\ \ \sigma(t_1)*\sigma(t_2)\\
				&\approx_B((s+1)*\sigma_1(t_1)+s*\sigma_2(t_1))*((s+1)*\sigma_1(t_2)+s*\sigma_2(t_2))\\
				&\approx_B(s+1)*\sigma_1(t_1)*\sigma_1(t_2)+0+0+s*\sigma_2(t_1)*\sigma_2(t_2)\\
				&=\ \ (s+1)*\sigma_1(t)+s*\sigma_2(t)\\
				\end{align*}
		\end{description}
		\end{proof}
		Now we can proof theorem \ref{lowenheim}. %TODO change text
		\begin{proof}[Proof(Theorem \ref{lowenheim})]
		At first we need to show that $\sigma$ is indeed a $B$-unifier of $t\approxq_B0$. This is easy since $\sigma$ has exactly the form we need for lemma \ref{lowenLemma} ($s:=t,\sigma_1(x):=x$ for all $x$ and $\sigma_2:=\tau$). The fact that $\tau$ is a $B$-unifier of $t\approxq_B0$ now gives us:
		\begin{align*}
		\sigma(t)&=(t+1)*t+t*\tau(t)
		\approx_B0+0
		\approx_B0
		\end{align*}
		So $\sigma$ is a $B$-unifier, it is easy to show that it is also reproductive. Let $\tau'$ be an arbitrary $B$-unifier of $t\approxq_B0$. If $x\in\mathcal{V}ar(t)$ then
		\begin{align*}
		\tau'(\sigma(x))&=\ \ \tau'((t+1)*x+t*\tau(x))\\
		&=\ \ (\tau'(t)+1)*\tau'(x)+\tau'(t)*\tau'(\tau(x))\\
		&\approx_B(0+1)*\tau'(x)+0*\tau'(\tau(x))\\
		&\approx_B\tau'(x)
		\end{align*}
		if $x\notin\mathcal{V}ar(t)$ then $\sigma(x)=x$ so $\tau'(\sigma(x))=\tau'(x)$ follows trivially.
		\end{proof}
		Now lets consider the $B$-unification problem $xy+yz+xz+1\approxq_B0$ (the same as in successive variable elimination) as example. We will only consider ground $B$-unifiers.
		\begin{align*}
		& x=1,y=1,z=1:& \sigma_1(x)&=\ \ (xy+yz+xz+1+1)*x+(xy+yz+xz+1)*1\\
		&&&\approx_Bxy+xyz+xz+(xy+yz+xz+1)\\
		&&&\approx_Bxyz+yz+1\\
		&&\sigma_1(y)&=\ \ (xy+yz+xz+1+1)*y+(xy+yz+xz+1)*1\\
		&&&\approx_Bxyz+xz+1\\
		&&\sigma_1(z)&=\ \ (xy+yz+xz+1+1)*z+(xy+yz+xz+1)*1\\
		&&&\approx_Bxyz+xy+1\\
		& x=0,y=1,z=1:& \sigma_2(x)&=\ \ (xy+yz+xz+1+1)*x+(xy+yz+xz+1)*0\\
		&&&\approx_Bxy+xyz+xz\\
		&&\sigma_2(y)&\approx_Bxyz+xz+1\\
		&&\sigma_2(z)&\approx_Bxyz+xy+1
		\end{align*}
		\begin{align*}
		& x=1,y=0,z=1\text{ and }x=1,y=1,z=0\text{ are symettric to }x=0,y=1,z=1
		\end{align*} 
		%Since it is not so easy to see that the calculated $B$-unifiers are reproductive we will consider some examples. First $\sigma_1(\sigma_2(w))\approx_B\sigma_1(w)$ for all $w$, if $w\notin\left\lbrace x,y,z\right\rbrace $ then $\sigma_1(\sigma_2(w))=\sigma_1(w)$ follows trivially. If $w\in\left\lbrace x,y,z\right\rbrace $ then
		%\begin{align*}
		%\sigma_1(\sigma_2(x))&\approx_B\sigma_1(xy+xyz+xz)\\
		%&\approx_B(xyz+yz+1)(xyz+xz+1)+\sigma_1(xyz+xz)\\
		%&\approx_B(xyz+xyz+xyz)+(xyz+xyz+xz)+(xyz+yz+1)+\sigma_1(xyz+xz)\\
		%&\approx_Bxz+yz+1+\sigma_1(xyz+xz)\\
		%&\approx_Bxz+yz+1+\sigma_1(xyz)+xy+yz+1\\
		%&\approx_Bxz+xy+(xz+yz+1)(xyz+xy+1)\\
		%&\approx_Bxz+xy+(xyz+xyz+xyz)+(xyz+xyz+xy)+(xz+yz+1)\\
		%&\approx_Bxyz+yz+1\\
		%&\approx_B\sigma_1(x)
		%\end{align*}
		Since it is not so easy to see that the calculated $B$-unifiers are reproductive we will consider an example.
		Remember the reproductive $B$-unifier $\sigma:=\left\lbrace x\mapsto xyz+yz+1,y\mapsto yz+z+1\right\rbrace$  we computed with successive variable elimination. Lets show that $\sigma(\sigma_1(w))\approx_B\sigma(w)$ for all $w$. If $w\notin\left\lbrace x,y,z\right\rbrace $ then $\sigma(\sigma_1(w))=\sigma(w)$ follows trivially. If $w\in\left\lbrace x,y,z\right\rbrace $ then
		\begin{align*}
		\sigma(\sigma_1(x))&\approx_B\sigma(xyz+yz+1)\\
		&\approx_B(xyz+yz+1)(yz+z+1)z+\sigma(yz+1)\\
		&\approx_B(xyz+yz+yz)+(xyz+yz+z)+(xyz+yz+z)+\sigma(yz+1)\\
		&\approx_Bxyz+(yz+z+1)z+1\\
		&\approx_Bxyz+yz+1\\
		&\approx_B\sigma(x)\\
		\sigma(\sigma_1(y))&\approx_B\sigma(xyz+xz+1)\\
		&\approx_Bxyz+\sigma(xz+1)\\
		&\approx_Bxyz+(xyz+yz+1)z+1\\
		&\approx_Byz+z+1\\
		&\approx_B\sigma(y)
		\end{align*}
		\begin{align*}
		\sigma(\sigma_1(z))&\approx_B\sigma(xyz+xy+1)\\
		&\approx_Bxyz+\sigma(xy+1)\\
		&\approx_Bxyz+(xyz+yz+1)(yz+z+1)+1\\
		&\approx_Bxyz+(xyz+yz+yz)+(xyz+yz+z)+(xyz+yz+1)+1\\
		&\approx_Bz\\
		&\approx_B\sigma(z)
		\end{align*}
		Of course this does not proof that $\sigma_1$ is reproductive it was just to stress the fact that for a reproductive $B$-unifier $\sigma$ the equation $\tau(\sigma(x))\approx_B\tau(x)$ has to hold for all $B$-unifiers $\tau$ so it especially has to hold for other reproductive $B$-unifiers.
		%TODO in Löwenheims formular in example show that the different mgus are reproductive (in \eqClass)