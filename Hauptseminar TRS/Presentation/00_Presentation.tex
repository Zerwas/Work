\documentclass{beamer}

\usepackage[utf8]{inputenc}
\usepackage[T1]{fontenc}

\usepackage{fixltx2e}
\usepackage{microtype}

\usepackage[style=british]{csquotes}

\usetheme[nogerman, nototalpages, noheader, nosectionnum,%
    einrichtung={Institute of Theoretical Computer Science\ },%
    professur={Chair of Automata Theory}]{TUDCD}

\setbeamerfont{block title}{size=\small}

\title{Unification modulo\hspace{30pt} Boolean Rings}

\author{Florian	Starke}

\date{\today}

%\datecity{Dresden}


\DefineNamedColor{named}{FgColour}{cmyk}{0,0.75,0.90,0}
\setbeamercolor{alerted text}{fg=FgColour}

\usepackage{amsmath}
\usepackage{amsfonts}
\usepackage{amssymb}
\usepackage{wasysym}
\usepackage{upgreek}

\usepackage{booktabs}
\usepackage{setspace}
\usepackage{xspace}
\usepackage{rotating}
\usepackage[square]{natbib}

\usepackage{mdframed}
\usepackage{tikz}
\usetikzlibrary{arrows}
\usetikzlibrary{automata}
\usetikzlibrary{positioning}

\hypersetup{%
    pdftitle={Title},
    pdfauthor={Author},
    pdfsubject={Title},
    pdfkeywords={Title},
    pdftoolbar=true,
    bookmarksopen=true,
    bookmarksnumbered=false,
    bookmarksopenlevel=1,
    hyperfootnotes=false,
    pdfdisplaydoctitle,
    plainpages=false,
    colorlinks=true,
    linkcolor=black,
    citecolor=black,
}

\newcommand{\approxq}{\stackrel{?}{\approx}}
\newcommand{\moreGeneral}{\lesssim^X_E}
\newcommand{\eqClass}{\sim^X_E}
\newcommand{\BTwo}{{\mathcal{B}_2}}
\newcommand{\PSet}{{\mathcal{P}_S}}
\newcommand{\TermSet}{T(\Sigma,V)}
\newcommand{\TermAlgebra}{\mathcal{T}(\Sigma,V)}

\begin{document}

\maketitle
\addtocounter{framenumber}{-1}
\title{Unification modulo Boolean Rings}

\section*{Structure}
\begin{frame}
\frametitle{Structure}
\tableofcontents 
\end{frame}
\section{Equational Unification}
\begin{frame}{Equational Unification}{Unification Problem}
\uncover<2->{$I:=\left\lbrace f(x,x)\approx x\right\rbrace$ }\vspace{15pt}
\begin{center}
$S:=\left\lbrace f(a,x)\stackrel{?}{=} x\right\rbrace\hspace{1cm}\uncover<3->{ \Rightarrow \hspace{1cm}S:=\left\lbrace f(a,x)\approxq_I x\right\rbrace}$
\end{center}
%$\sigma:=\left\lbrace x\mapsto a \right\rbrace $
\end{frame}
\begin{frame}{Equational Unification}{Unification Classes}
Let $S$ be an $I$-unification problem \dots

\begin{description}
\setlength{\itemsep}{15pt}
\item[\textbf{elementary}:] $S:=\left\lbrace f(y,x)\approxq_I x\right\rbrace$
\item[with \textbf{constants}:] $S:=\left\lbrace f(a,x)\approxq_I x\right\rbrace$
\item[\textbf{general}:] $S:=\left\lbrace f(g(a),x)\approxq_I x\right\rbrace$
\end{description}
\end{frame}

\begin{frame}{Equational Unification}{MGU?}
$C:=\left\lbrace f(x,y)\approx f(y,x)\right\rbrace$
\begin{align*}
S&:=\left\lbrace f(x,y)\approxq_Cf(a,b)\right\rbrace 
\end{align*}
\uncover<2->{
\begin{align*}
\sigma_1&:=\left\lbrace x\mapsto a,y\mapsto b\right\rbrace& \sigma_2&:=\left\lbrace x\mapsto b,y\mapsto a\right\rbrace
\end{align*}
}
\end{frame}

\begin{frame}{Equational Unification}{More General}
\begin{center}
$
\def\arraystretch{1.5}
\arraycolsep=20pt
\begin{array}{ccc}
\sigma\leq\sigma'&&\uncover<2->{\sigma\moreGeneral\sigma'}\\
\textbf{iff}&\uncover<2->{\Rightarrow&\hspace{5pt}\textbf{iff}}\\
\exists\delta:\delta\sigma=\sigma'&&\uncover<2->{\exists\delta:\forall x\in X:\\
&&\delta(\sigma(x))\approx_E\sigma'(x)}
\end{array}
$
\end{center}
\end{frame}

\begin{frame}{Equational Unification}{Minimal Complete Sets}
%\frametitle{Minimal Complete Sets}
Let $S$ be an $E$-unification problem. A \textbf{minimal complete} set for $S$ is a set of substitutions $\mathcal{M}$ that satisfy the following properties:

\begin{itemize}
\setlength{\itemsep}{9pt}
	\item each $\sigma \in \mathcal{M}$ is an $E$-unifier of $S$
	\item for all $\theta\in\mathcal{U}_E(S)$ there exists a $\sigma \in \mathcal{M}$ such that $\sigma\moreGeneral\theta$
	\item for all $\sigma,\sigma'\in\mathcal{M},\ \sigma\moreGeneral\sigma'$ implies $\sigma=\sigma'$.
\end{itemize}
\end{frame}
%minimal complete sets have same cardinality

\begin{frame}{Equational Unification}{Unification Type of $\approx_E$}
	\begin{description}
	\setlength{\itemsep}{7pt}
		\item[\textbf{unitary}] iff for all $E$-unification problems $S$ there exists a minimal complete set of cardinality $\leq 1$.
		%if S has no solution M=\emptyset
		\uncover<2->{
		\item[\textbf{finitary}] iff for all $E$-unification problems $S$ there exists a minimal complete set with finite cardinality.}
		\uncover<3->{
		\item[\textbf{infinitary}] iff for all $E$-unification problems $S$ there exists a minimal complete set, and there exists an $E$-unification problem for which this set is infinite.}
		\uncover<4->{
		\item[\textbf{zero}] iff there exists an $E$-unification problem that does not have a minimal complete set.}
	\end{description}
\end{frame}

\section{Boolean Rings}
\begin{frame}{Boolean Rings}
\[B:=\left\lbrace 
	\begin{aligned}
		x+y     & \approx y+x,         & x*y     & \approx y*x,     \\
		(x+y)+z & \approx x+(y+z),     & (x*y)*z & \approx x*(y*z), \\
		x+x     & \approx 0,           & x*x     & \approx x,       \\
		0+x     & \approx x,           & 0*x     & \approx 0,       \\
		x*(y+z) & \approx (x*y)+(x*z), & 1*x     & \approx x        
	\end{aligned}
	\right\rbrace \]
\end{frame}

\begin{frame}{Boolean Rings}{Interpretation}
$\BTwo:=(\Delta^\BTwo,\cdot^\BTwo)$\\
\vspace{10pt}\hspace{10pt}
$\Delta^\BTwo:=\left\lbrace\bot,\top\right\rbrace$
\uncover<2->{
\begin{align*}
	(x+y)^\BTwo & :=\left( x^\BTwo\wedge\neg y^\BTwo\right)  \vee \left( \neg x^\BTwo\wedge y^\BTwo\right) \\ 
	(x*y)^\BTwo & :=x^\BTwo\wedge y^\BTwo \\
	0^\BTwo     & :=\bot \\ 
	1^\BTwo     & :=\top                  
\end{align*}
}
\end{frame}

\begin{frame}{Boolean Rings}{Example}	
\begin{align*}
	(1+0)^\BTwo&=\left( 1^\BTwo\wedge\neg 0^\BTwo\right) \vee \left( \neg1^\BTwo\wedge 0^\BTwo\right) \\
	&=\left( \top\wedge\neg\bot\right) \vee\left(\neg\top\wedge\bot \right) \\
	&=\top\vee\bot\\
	&=\top
\end{align*}
\end{frame}


\section{Unification modulo Boolean Rings}
%into slide elementary(!) BR-Unification 
\begin{frame}{Unification modulo Boolean Rings}
In the following we will only consider elementary $B$-unification.\vspace{10pt}

We will see:
\begin{itemize}
\setlength{\itemsep}{6pt}
\item how to find a $B$-unifier.
\item how to turn this $B$-unifier into an mgu.
\item that elementary $B$-unification is unitary.
\end{itemize}
\end{frame}

%how to turn a solution in \BTwo into a $B$-unifier
\begin{frame}{Unification modulo Boolean Rings}{Solution in $\BTwo\Rightarrow B$-unifier}
$S:=x+y+z\approxq_B z+1$\vspace{15pt}
\uncover<2->{
\begin{center}
$
\arraycolsep=10pt
\begin{array}{ccc}
      	\varphi(w):=\begin{cases}
      	\bot & \text{if }w=x     \\
      	\top & \text{if }w\neq x 
      	\end{cases}&\uncover<3->{\Rightarrow&\sigma':=\left\lbrace x\mapsto0, y\mapsto1,z\mapsto 1\right\rbrace}
\end{array}
$
\end{center}
}
\end{frame}

%t=0
\begin{frame}{Unification modulo Boolean Rings}{Transformation}
\begin{align*}
&	& S&=\left\lbrace s_1\approxq_B t_1,\dots, s_n\approxq_B t_n\right\rbrace\\
\uncover<2->{&\Rightarrow & S&=\left\lbrace s_1+t_1\approxq_B 0,\dots, s_n+t_n\approxq_B 0\right\rbrace\\}
\uncover<3->{&\Rightarrow & S&=\left\lbrace (s_1+t_1+1)*\dots*(s_n+t_n+1)\approxq_B 1\right\rbrace\\}
\uncover<4->{&\Rightarrow & S&=\left\lbrace (s_1+t_1+1)*\dots*(s_n+t_n+1)+1\approxq_B 0\right\rbrace}
\end{align*}
\end{frame}

\begin{frame}{Unification modulo Boolean Rings}{Reproductive $E$-unifier}
\begin{center}
$
\def\arraystretch{1.5}
\arraycolsep=10pt
\begin{array}{ccc}
\sigma \text{ is a most general $E$-unifier of }S&&\uncover<2->{\sigma\textbf{ is a reproductive $E$-unifier of }S}\\
\text{iff}&\uncover<2->{\Rightarrow&\textbf{iff}}\\
\forall \tau\in\mathcal{U}_E(S):\exists\theta:\forall x\in X:&&\uncover<2->{\forall \tau\in\mathcal{U}_E(S):\forall x:}\\
\theta(\sigma(x))\approx_E\tau(x)&&\uncover<2->{\tau(\sigma(x))\approx_E\tau(x)}
\end{array}
$
\end{center}
\end{frame}

\begin{frame}{Unification modulo Boolean Rings}{Löwenheim's Formula}
Let $\tau$ be a $B$-unifier of $t\approxq_B0$. The substitution $\sigma$ defined by
\begin{align*}
	\sigma(x):=\begin{cases}
	(t+1)*x+t*\tau(x) & \text{if }x\in\mathcal{V}ar(t)    \\
	x                 & \text{if }x\notin\mathcal{V}ar(t) 
	\end{cases}
\end{align*}
is a reproductive $B$-unifier of $t\approxq_B0$.
\end{frame}

\begin{frame}{Unification modulo Boolean Rings}{Löwenheim's Formula}
Why is $\sigma$ reproductive?\vspace{10pt}

Let $\tau'$ be an arbitrary $B$-unifier of $S$.
\begin{align*}
\tau'(\sigma(x)) & =\ \ \tau'((t+1)*x+t*\tau(x))                      \\
        & =\ \ (\tau'(t)+1)*\tau'(x)+\tau'(t)*\tau'(\tau(x)) \\
        & \approx_B(0+1)*\tau'(x)+0*\tau'(\tau(x))           \\
        & \approx_B\tau'(x)                                  
\end{align*}
\end{frame}

\begin{frame}{Unification modulo Boolean Rings}{Löwenheim's Formula Example}
$B$-unification problem: $xy\approxq_B0$\vspace{10pt}
$
\def\arraystretch{1.3}
\begin{array}{ll}
 x=0,y=0:	&\uncover<2->{\sigma_1(x)=(xy+1)*x+xy*0\approx_B xy+x\\
			&\sigma_1(y)=(xy+1)*y+xy*0\approx_B xy+y\\}
 x=0,y=1:	&\uncover<3->{\sigma_2(x)=(xy+1)*x+xy*0\approx_B xy+x\\
			&\sigma_2(y)=(xy+1)*y+xy*1\approx_B y\\}
 x=1,y=0:	&\uncover<3->{ \text{similar to }x=0,y=1.}
\end{array}$
\end{frame}

%summary
\begin{frame}{Unification modulo Boolean Rings}{Summary}
\begin{itemize}
\setlength{\itemsep}{20pt}
\item[] Equational unification needs minimal complete sets of unifiers.
\item[] Elementary $B$-unification is unitary.
\item[] Finding an mgu is NP-complete.
\end{itemize}
\end{frame}
\section*{end}
\end{document}