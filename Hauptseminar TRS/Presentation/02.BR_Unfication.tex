\section{Boolean Rings}
\begin{frame}{Boolean Rings}
\[B:=\left\lbrace 
	\begin{aligned}
		x+y     & \approx y+x,         & x*y     & \approx y*x,     \\
		(x+y)+z & \approx x+(y+z),     & (x*y)*z & \approx x*(y*z), \\
		x+x     & \approx 0,           & x*x     & \approx x,       \\
		0+x     & \approx x,           & 0*x     & \approx 0,       \\
		x*(y+z) & \approx (x*y)+(x*z), & 1*x     & \approx x        
	\end{aligned}
	\right\rbrace \]
\end{frame}

\begin{frame}{Boolean Rings}{Interpretation}
$\BTwo:=(\Delta^\BTwo,\cdot^\BTwo)$\\
\vspace{10pt}\hspace{10pt}
$\Delta^\BTwo:=\left\lbrace\bot,\top\right\rbrace$
\uncover<2->{
\begin{align*}
	(x+y)^\BTwo & :=\left( x^\BTwo\wedge\neg y^\BTwo\right)  \vee \left( \neg x^\BTwo\wedge y^\BTwo\right) \\ 
	(x*y)^\BTwo & :=x^\BTwo\wedge y^\BTwo \\
	0^\BTwo     & :=\bot \\ 
	1^\BTwo     & :=\top                  
\end{align*}
}
\end{frame}

\begin{frame}{Boolean Rings}{Example}	
\begin{align*}
	(1+0)^\BTwo&=\left( 1^\BTwo\wedge\neg 0^\BTwo\right) \vee \left( \neg1^\BTwo\wedge 0^\BTwo\right) \\
	&=\left( \top\wedge\neg\bot\right) \vee\left(\neg\top\wedge\bot \right) \\
	&=\top\vee\bot\\
	&=\top
\end{align*}
\end{frame}


\section{Unification modulo Boolean Rings}
%into slide elementary(!) BR-Unification 
\begin{frame}{Unification modulo Boolean Rings}
In the following we will only consider elementary $B$-unification.\vspace{10pt}

We will see:
\begin{itemize}
\setlength{\itemsep}{6pt}
\item how to find a $B$-unifier.
\item how to turn this $B$-unifier into an mgu.
\item that elementary $B$-unification is unitary.
\end{itemize}
\end{frame}

%how to turn a solution in \BTwo into a $B$-unifier
\begin{frame}{Unification modulo Boolean Rings}{Solution in $\BTwo\Rightarrow B$-unifier}
$S:=x+y+z\approxq_B z+1$\vspace{15pt}
\uncover<2->{
\begin{center}
$
\arraycolsep=10pt
\begin{array}{ccc}
      	\varphi(w):=\begin{cases}
      	\bot & \text{if }w=x     \\
      	\top & \text{if }w\neq x 
      	\end{cases}&\uncover<3->{\Rightarrow&\sigma':=\left\lbrace x\mapsto0, y\mapsto1,z\mapsto 1\right\rbrace}
\end{array}
$
\end{center}
}
\end{frame}

%t=0
\begin{frame}{Unification modulo Boolean Rings}{Transformation}
\begin{align*}
&	& S&=\left\lbrace s_1\approxq_B t_1,\dots, s_n\approxq_B t_n\right\rbrace\\
\uncover<2->{&\Rightarrow & S&=\left\lbrace s_1+t_1\approxq_B 0,\dots, s_n+t_n\approxq_B 0\right\rbrace\\}
\uncover<3->{&\Rightarrow & S&=\left\lbrace (s_1+t_1+1)*\dots*(s_n+t_n+1)\approxq_B 1\right\rbrace\\}
\uncover<4->{&\Rightarrow & S&=\left\lbrace (s_1+t_1+1)*\dots*(s_n+t_n+1)+1\approxq_B 0\right\rbrace}
\end{align*}
\end{frame}

\begin{frame}{Unification modulo Boolean Rings}{Reproductive $E$-unifier}
\begin{center}
$
\def\arraystretch{1.5}
\arraycolsep=10pt
\begin{array}{ccc}
\sigma \text{ is an mgu of }S&&\uncover<2->{\sigma\textbf{ is a reproductive $E$-unifier of }S}\\
\text{iff}&\uncover<2->{\Rightarrow&\textbf{iff}}\\
\forall \tau\in\mathcal{U}_E(S):\exists\theta:\forall x\in X:&&\uncover<2->{\forall \tau\in\mathcal{U}_E(S):\forall x:}\\
\theta(\sigma(x))\approx_E\tau(x)&&\uncover<2->{\tau(\sigma(x))\approx_E\tau(x)}
\end{array}
$
\end{center}
\end{frame}

\begin{frame}{Unification modulo Boolean Rings}{Löwenheim's Formula}
Let $\tau$ be a $B$-unifier of $t\approxq_B0$. The substitution $\sigma$ defined by
\begin{align*}
	\sigma(x):=\begin{cases}
	(t+1)*x+t*\tau(x) & \text{if }x\in\mathcal{V}ar(t)    \\
	x                 & \text{if }x\notin\mathcal{V}ar(t) 
	\end{cases}
\end{align*}
is a reproductive $B$-unifier of $t\approxq_B0$.
\end{frame}

\begin{frame}{Unification modulo Boolean Rings}{Löwenheim's Formula}
Why is $\sigma$ reproductive?\vspace{10pt}

Let $\tau'$ be an arbitrary $B$-unifier of $S$.
\begin{align*}
\tau'(\sigma(x)) & =\ \ \tau'((t+1)*x+t*\tau(x))                      \\
        & =\ \ (\tau'(t)+1)*\tau'(x)+\tau'(t)*\tau'(\tau(x)) \\
        & \approx_B(0+1)*\tau'(x)+0*\tau'(\tau(x))           \\
        & \approx_B\tau'(x)                                  
\end{align*}
\end{frame}

\begin{frame}{Unification modulo Boolean Rings}{Löwenheim's Formula Example}
$B$-unification problem: $xy\approxq_B0$\vspace{10pt}
$
\def\arraystretch{1.3}
\begin{array}{ll}
 x=0,y=0:	&\uncover<2->{\sigma_1(x)=(xy+1)*x+xy*0\approx_B xy+x\\
			&\sigma_1(y)=(xy+1)*y+xy*0\approx_B xy+y\\}
 x=0,y=1:	&\uncover<3->{\sigma_2(x)=(xy+1)*x+xy*0\approx_B xy+x\\
			&\sigma_2(y)=(xy+1)*y+xy*1\approx_B y\\}
 x=1,y=0:	&\uncover<3->{ \text{similar to }x=0,y=1.}
\end{array}$
\end{frame}

%summary
\begin{frame}{Unification modulo Boolean Rings}{Summary}
\begin{itemize}
\setlength{\itemsep}{20pt}
\item[] Equational unification needs minimal complete sets of unifiers.
\item[] Elementary $B$-unification is unitary.
\item[] Finding an mgu is NP-complete.
\end{itemize}
\end{frame}