\documentclass[]{scrartcl}
\usepackage[utf8]{inputenc}
\usepackage{amsthm}
\usepackage{amssymb}
\usepackage{amsmath}
\usepackage{calc}
\usepackage{graphicx}
\usepackage[ngerman, german]{babel}
\usepackage{caption}
\usepackage{array}
\usepackage{float}

\makeatletter
\newcommand{\thickhline}{%
    \noalign {\ifnum 0=`}\fi \hrule height 1pt
    \futurelet \reserved@a \@xhline
}
\makeatother


\newcommand{\R}{\mathbb{R}}
\newcommand{\N}{\mathbb{N}}

\title{Modulbegleitende Aufgabe II}
\author{Shanshan Huang, Florian Starke}
\begin{document}
\def\smooth{_smooth}
	\maketitle
	Gegeben seien $N\in\N$, eine Zerlegung $\Delta_N$ des Intervalls $[-1,1]$ durch die Stützstellen $-1\leq x_0\leq x_1\leq\dots\leq x_N\leq1$, und die Funktionen $f_R,f_1\colon\R\to\R$ mit
	\[f_R(x):=\frac{1}{1+25x^2},\]
	\[f_1(x):=(1+\cos(\frac{3}{2}\pi x))^{2/3}.\]
	\section{Polynominterpolation}
	\subsection{Gleichverteilte Stützstellen}
	Die $N+1$ Stützstellen sind äquidistant verteilt. Es folgt $x_i:=-1+2i/N$ für $i=0,\dots,N$.
	\begin{figure}[H]
		\centering
		\begin{minipage}{0.5\textwidth}
			\includegraphics[width=8cm,keepaspectratio]{runge_aequidistant\smooth}
			\caption{\label{Abb.1}}
		\end{minipage}
		\begin{minipage}{0.49\textwidth}
			\includegraphics[width=8cm,keepaspectratio]{runge_err\smooth}
			\caption{\label{Abb.2}}
		\end{minipage}
	\end{figure}
	
	In Abbildung \ref{Abb.1} ist $f_R$ und das interpolierte Polynom $g_{12}$ im Intervall $I$ abgebildet. Wie erwartet ist bei einer Gleichverteilung der Stützstellen der Fehler am Rand sehr groß.
	
	\begin{figure}[H]
		\centering
		\begin{minipage}{0.5\textwidth}
			\includegraphics[width=8cm,keepaspectratio]{f1_aequidistant\smooth}
			\caption{\label{Abb.f1.1}}
		\end{minipage}
		\begin{minipage}{0.49\textwidth}
			\includegraphics[width=8cm,keepaspectratio]{f1_err\smooth}
			\caption{\label{Abb.f1.2}}
		\end{minipage}
	\end{figure}
	In den Abbildungen \ref{Abb.f1.1} und \ref{Abb.f1.2} sieht man die Funktion $f_1$ zusammen mit entsprechendem $g_{12}$ bzw. $F_{12}$.
	
	
	Begründung des Fehlerverhaltens:
	
	Wenn wir mehr Stützstellen hinzufügen, kann es sein ,dass die Polynomfunktion im Allgemeinen die zu interpolierenden Funktion nicht besser approximiert. Da der Interpolationsfehler wie folgt abgeschätzt werden kann.
	
	\[\lVert f(x)-g_N(x) \rVert\leq\frac{\lVert f^{(n+1)}\rVert}{(n+1)!}\lVert\omega\rVert\]
	
	Je höher der Grad ist, desto größer ist der Fehler, da bei der Runge Funktion die Ableitungen sehr groß werden. Der Fehler ist an den Intervallgrenzen am größten, da hier $|\omega|$ maximal ist. 
	
	\subsection{Tschebyschow-Stützstellen}
	Als Stützstellen werden die Nullstellen des Tschebyschow-Polynoms $T_{N+1}$ gewählt. Also definieren wir $x_i:=\cos(\frac{2i+1}{2N+2}\pi)$ für $i=0,\dots,N$.

	\newpage
	Die Runge Funktion und $g_{12}$:
	\begin{figure}[H]
		\centering
		\begin{minipage}{0.5\textwidth}
			\includegraphics[width=8cm,keepaspectratio]{runge_tscheb\smooth}
			\caption{\label{Abb.3}}
		\end{minipage}
		\begin{minipage}{0.49\textwidth}
			\includegraphics[width=8cm,keepaspectratio]{runge_tscheb_err\smooth}
			\caption{\label{Abb.4}}
		\end{minipage}
	\end{figure}
	
	Wie erwartet ist der Fehler an den Intervallgrenzen wesentlich geringer als bei äquidistanten Stützstellen.
	
	Die Funktion $f_1$:
	\begin{figure}[H]
		\centering
		\begin{minipage}{0.5\textwidth}
			\includegraphics[width=8cm,keepaspectratio]{f1_tscheb\smooth}
			\caption{}
		\end{minipage}
		\begin{minipage}{0.49\textwidth}
			\includegraphics[width=8cm,keepaspectratio]{f1_tscheb_err\smooth}
			\caption{}
		\end{minipage}
	\end{figure}
	
	Begründung des Fehlerverhaltens:
	
	Bei Tschebyschow Stützstellen ist $\omega$ am Rand kleiner, dafür aber in der Intervallmittel größer, als bei äquidistanten Stützstellen. Dementsprechend auch der Fehler.
	
	\section{Spline-Interpolation}
	
	Ziel ist es jetzt nicht mehr die Funktion $f$ durch ein Polynom zu interpolieren sondern nur noch durch eine stückweise polynomielle Funktion (Spline). In diesem Fall geht es um Splines vom Grad 3 mit Glattheit 1. Wir kennen sowohl die Funktion als auch die erste Ableitung der Funktion.
	
	Sei $s$ die gesuchte Spline Funktion. Dann ist $s_k:=s|_{[x_k,x_{k+1}]}$ (für $k=0,\dots,N-1$) ein Polynom dritten Grades mit $s_k=a_k(x-x_k)^3 + b_k(x-x_k)^2 + c_k(x-x_k) + d_k$. Wobei die Koeffizienten aus den gegebenen Funktions- und Ableitungswerten wie folgt berechnet werden können.
	
	\begin{align*}
		a_k&=\frac{-2}{h_k^3}\cdot(f(x_{k+1})-f(x_k)) + \frac{1}{h_k^2}\cdot(f'(x_k)+f'(x_{k+1}))\\
		b_k&=\frac{3}{h_k^2}\cdot(f(x_{k+1})-f(x_k)) - \frac{1}{h_k} \cdot(2f'(x_k)+f'(x_{k+1}))\\
		c_k&=f'(x_k)\\
		d_k&=f(x_k)
	\end{align*}
	Mit $h_k:=x_{k+1}-x_k$. Für Herleitung siehe Vorlesung.
	
	Die erste Ableitung der Runge Funktion ist:
	
	\[f_R'(x)=\frac{-50x}{(25x^2+1)^2}\]
	
	
	\begin{figure}[H]
		\centering
		\begin{minipage}{0.5\textwidth}
			\includegraphics[width=8cm,keepaspectratio]{runge_spline\smooth}
			\caption{\label{Abb.5}}
		\end{minipage}
		\begin{minipage}{0.49\textwidth}
			\includegraphics[width=8cm,keepaspectratio]{runge_spline_err\smooth}
			\caption{\label{Abb.6}}
		\end{minipage}
	\end{figure}
	
	Wie man sieht ist die Spline Interpolation wesentlich genauer als die Polynominterpolation.
	
	\newpage
	
	Fehler für $N_1=2$, $N_2=4$, und $N_3=8$.
 	
 	\begin{figure}[H]
 		\centering
		\includegraphics[width=8cm,keepaspectratio]{runge_spline_err_123\smooth}
		\caption{\label{Abb.9}}
 	\end{figure}
 	
 	\renewcommand{\arraystretch}{1.2}
 	\begin{tabular}{|c|c|c|}\hline
 	$k$  & $E(h_{N_k})$ & $\operatorname{EOC}(h_{N_k},h_{N_{k+1}})$\\\thickhline
 	$1$  & $4.8928\times10^{-1}$ & $1.1572$\\\hline
 	$2$  & $2.1938\times10^{-1}$ & $2.6272$\\\hline
 	$3$  & $3.5509\times10^{-2}$ & $4.3901$\\\hline
 	$4$  & $1.6935\times10^{-3}$ & $2.1237$\\\hline
 	$5$  & $3.8860\times10^{-4}$ & $3.5334$\\\hline
 	$6$  & $3.3560\times10^{-5}$ & $3.8869$\\\hline
 	$7$  & $2.2686\times10^{-6}$ & $3.9719$\\\hline
 	$8$  & $1.4917\times10^{-7}$ & $3.9930$\\\hline
 	$9$  & $9.0802\times10^{-9}$ & $3.9982$\\\hline
 	$10$ & $5.6820\times10^{-10}$ & $3.9996$\\\hline
 	$11$ & $3.5523\times10^{-11}$ & $3.9999$\\\hline
 	$12$ & $2.2204\times10^{-12}$ & $-$\\\hline
 	\end{tabular}
 	
 	Nochmal Spline-Interpolation für $f_1$:
	\begin{figure}[h]
		\centering
		\begin{minipage}{0.5\textwidth}
			\includegraphics[width=8cm,keepaspectratio]{f1_spline\smooth}
			\caption{}
		\end{minipage}
		\begin{minipage}{0.49\textwidth}
			\includegraphics[width=8cm,keepaspectratio]{f1_spline_err\smooth}
			\caption{}
		\end{minipage}
	\end{figure} 
	
	Fehler für $N_1=2$, $N_2=4$, und $N_3=8$.
 	
 	\begin{figure}[H]
 		\centering
		\includegraphics[width=8cm,keepaspectratio]{f1_spline_err_123\smooth}
		\caption{}
 	\end{figure}
\end{document}
